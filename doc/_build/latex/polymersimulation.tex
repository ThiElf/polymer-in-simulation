%% Generated by Sphinx.
\def\sphinxdocclass{report}
\documentclass[letterpaper,10pt,english]{sphinxmanual}
\ifdefined\pdfpxdimen
   \let\sphinxpxdimen\pdfpxdimen\else\newdimen\sphinxpxdimen
\fi \sphinxpxdimen=.75bp\relax

\PassOptionsToPackage{warn}{textcomp}
\usepackage[utf8]{inputenc}
\ifdefined\DeclareUnicodeCharacter
% support both utf8 and utf8x syntaxes
  \ifdefined\DeclareUnicodeCharacterAsOptional
    \def\sphinxDUC#1{\DeclareUnicodeCharacter{"#1}}
  \else
    \let\sphinxDUC\DeclareUnicodeCharacter
  \fi
  \sphinxDUC{00A0}{\nobreakspace}
  \sphinxDUC{2500}{\sphinxunichar{2500}}
  \sphinxDUC{2502}{\sphinxunichar{2502}}
  \sphinxDUC{2514}{\sphinxunichar{2514}}
  \sphinxDUC{251C}{\sphinxunichar{251C}}
  \sphinxDUC{2572}{\textbackslash}
\fi
\usepackage{cmap}
\usepackage[T1]{fontenc}
\usepackage{amsmath,amssymb,amstext}
\usepackage{babel}



\usepackage{times}
\expandafter\ifx\csname T@LGR\endcsname\relax
\else
% LGR was declared as font encoding
  \substitutefont{LGR}{\rmdefault}{cmr}
  \substitutefont{LGR}{\sfdefault}{cmss}
  \substitutefont{LGR}{\ttdefault}{cmtt}
\fi
\expandafter\ifx\csname T@X2\endcsname\relax
  \expandafter\ifx\csname T@T2A\endcsname\relax
  \else
  % T2A was declared as font encoding
    \substitutefont{T2A}{\rmdefault}{cmr}
    \substitutefont{T2A}{\sfdefault}{cmss}
    \substitutefont{T2A}{\ttdefault}{cmtt}
  \fi
\else
% X2 was declared as font encoding
  \substitutefont{X2}{\rmdefault}{cmr}
  \substitutefont{X2}{\sfdefault}{cmss}
  \substitutefont{X2}{\ttdefault}{cmtt}
\fi


\usepackage[Bjarne]{fncychap}
\usepackage{sphinx}

\fvset{fontsize=\small}
\usepackage{geometry}


% Include hyperref last.
\usepackage{hyperref}
% Fix anchor placement for figures with captions.
\usepackage{hypcap}% it must be loaded after hyperref.
% Set up styles of URL: it should be placed after hyperref.
\urlstyle{same}

\usepackage{sphinxmessages}
\setcounter{tocdepth}{0}



\title{Polymer Simulation}
\date{Jul 21, 2020}
\release{}
\author{Thiago P.\@{} O.\@{} Nogueira}
\newcommand{\sphinxlogo}{\vbox{}}
\renewcommand{\releasename}{}
\makeindex
\begin{document}

\pagestyle{empty}
\sphinxmaketitle
\pagestyle{plain}
\sphinxtableofcontents
\pagestyle{normal}
\phantomsection\label{\detokenize{index::doc}}



\chapter{Getting started}
\label{\detokenize{getting_started:getting-started}}\label{\detokenize{getting_started:id1}}\label{\detokenize{getting_started::doc}}

\section{Installing your doc directory}
\label{\detokenize{getting_started:installing-your-doc-directory}}\label{\detokenize{getting_started:installing-docdir}}
You may already have sphinx \sphinxhref{http://sphinx.pocoo.org/}{sphinx}
installed \textendash{} you can check by doing:

\begin{sphinxVerbatim}[commandchars=\\\{\}]
\PYG{n}{python} \PYG{o}{\PYGZhy{}}\PYG{n}{c} \PYG{l+s+s1}{\PYGZsq{}}\PYG{l+s+s1}{import sphinx}\PYG{l+s+s1}{\PYGZsq{}}
\end{sphinxVerbatim}

If that fails grab the latest version of and install it with:

\begin{sphinxVerbatim}[commandchars=\\\{\}]
\PYG{o}{\PYGZgt{}} \PYG{n}{sudo} \PYG{n}{easy\PYGZus{}install} \PYG{o}{\PYGZhy{}}\PYG{n}{U} \PYG{n}{Sphinx}
\end{sphinxVerbatim}

Now you are ready to build a template for your docs, using
sphinx\sphinxhyphen{}quickstart:

\begin{sphinxVerbatim}[commandchars=\\\{\}]
\PYG{o}{\PYGZgt{}} \PYG{n}{sphinx}\PYG{o}{\PYGZhy{}}\PYG{n}{quickstart}
\end{sphinxVerbatim}

accepting most of the defaults.  I choose “sampledoc” as the name of my
project.  cd into your new directory and check the contents:

\begin{sphinxVerbatim}[commandchars=\\\{\}]
\PYG{n}{home}\PYG{p}{:}\PYG{o}{\PYGZti{}}\PYG{o}{/}\PYG{n}{tmp}\PYG{o}{/}\PYG{n}{sampledoc}\PYG{o}{\PYGZgt{}} \PYG{n}{ls}
\PYG{n}{Makefile}      \PYG{n}{\PYGZus{}static}         \PYG{n}{conf}\PYG{o}{.}\PYG{n}{py}
\PYG{n}{\PYGZus{}build}                \PYG{n}{\PYGZus{}templates}      \PYG{n}{index}\PYG{o}{.}\PYG{n}{rst}
\end{sphinxVerbatim}

The index.rst is the master ReST for your project, but before adding
anything, let’s see if we can build some html:

\begin{sphinxVerbatim}[commandchars=\\\{\}]
\PYG{n}{make} \PYG{n}{html}
\end{sphinxVerbatim}

If you now point your browser to \sphinxcode{\sphinxupquote{\_build/html/index.html}}, you
should see a basic sphinx site.

\noindent\sphinxincludegraphics{{basic_screenshot}.png}


\subsection{Fetching the data}
\label{\detokenize{getting_started:fetching-the-data}}\label{\detokenize{getting_started:id2}}
Now we will start to customize out docs.  Grab a couple of files from
the \sphinxhref{https://github.com/matplotlib/sampledoc}{web site}
or git.  You will need \sphinxcode{\sphinxupquote{getting\_started.rst}} and
\sphinxcode{\sphinxupquote{\_static/basic\_screenshot.png}}.  All of the files live in the
“completed” version of this tutorial, but since this is a tutorial,
we’ll just grab them one at a time, so you can learn what needs to be
changed where.  Since we have more files to come, I’m going to grab
the whole git directory and just copy the files I need over for now.
First, I’ll cd up back into the directory containing my project, check
out the “finished” product from git, and then copy in just the files I
need into my \sphinxcode{\sphinxupquote{sampledoc}} directory:

\begin{sphinxVerbatim}[commandchars=\\\{\}]
\PYG{n}{home}\PYG{p}{:}\PYG{o}{\PYGZti{}}\PYG{o}{/}\PYG{n}{tmp}\PYG{o}{/}\PYG{n}{sampledoc}\PYG{o}{\PYGZgt{}} \PYG{n}{pwd}
\PYG{o}{/}\PYG{n}{Users}\PYG{o}{/}\PYG{n}{jdhunter}\PYG{o}{/}\PYG{n}{tmp}\PYG{o}{/}\PYG{n}{sampledoc}
\PYG{n}{home}\PYG{p}{:}\PYG{o}{\PYGZti{}}\PYG{o}{/}\PYG{n}{tmp}\PYG{o}{/}\PYG{n}{sampledoc}\PYG{o}{\PYGZgt{}} \PYG{n}{cd} \PYG{o}{.}\PYG{o}{.}
\PYG{n}{home}\PYG{p}{:}\PYG{o}{\PYGZti{}}\PYG{o}{/}\PYG{n}{tmp}\PYG{o}{\PYGZgt{}} \PYG{n}{git} \PYG{n}{clone} \PYG{n}{https}\PYG{p}{:}\PYG{o}{/}\PYG{o}{/}\PYG{n}{github}\PYG{o}{.}\PYG{n}{com}\PYG{o}{/}\PYG{n}{matplotlib}\PYG{o}{/}\PYG{n}{sampledoc}\PYG{o}{.}\PYG{n}{git} \PYG{n}{tutorial}
\PYG{n}{Cloning} \PYG{n}{into} \PYG{l+s+s1}{\PYGZsq{}}\PYG{l+s+s1}{tutorial}\PYG{l+s+s1}{\PYGZsq{}}\PYG{o}{.}\PYG{o}{.}\PYG{o}{.}
\PYG{n}{remote}\PYG{p}{:} \PYG{n}{Counting} \PYG{n}{objects}\PYG{p}{:} \PYG{l+m+mi}{87}\PYG{p}{,} \PYG{n}{done}\PYG{o}{.}
\PYG{n}{remote}\PYG{p}{:} \PYG{n}{Compressing} \PYG{n}{objects}\PYG{p}{:} \PYG{l+m+mi}{100}\PYG{o}{\PYGZpc{}} \PYG{p}{(}\PYG{l+m+mi}{43}\PYG{o}{/}\PYG{l+m+mi}{43}\PYG{p}{)}\PYG{p}{,} \PYG{n}{done}\PYG{o}{.}
\PYG{n}{remote}\PYG{p}{:} \PYG{n}{Total} \PYG{l+m+mi}{87} \PYG{p}{(}\PYG{n}{delta} \PYG{l+m+mi}{45}\PYG{p}{)}\PYG{p}{,} \PYG{n}{reused} \PYG{l+m+mi}{83} \PYG{p}{(}\PYG{n}{delta} \PYG{l+m+mi}{41}\PYG{p}{)}
\PYG{n}{Unpacking} \PYG{n}{objects}\PYG{p}{:} \PYG{l+m+mi}{100}\PYG{o}{\PYGZpc{}} \PYG{p}{(}\PYG{l+m+mi}{87}\PYG{o}{/}\PYG{l+m+mi}{87}\PYG{p}{)}\PYG{p}{,} \PYG{n}{done}\PYG{o}{.}
\PYG{n}{Checking} \PYG{n}{connectivity}\PYG{o}{.}\PYG{o}{.}\PYG{o}{.} \PYG{n}{done}
\PYG{n}{home}\PYG{p}{:}\PYG{o}{\PYGZti{}}\PYG{o}{/}\PYG{n}{tmp}\PYG{o}{\PYGZgt{}} \PYG{n}{cp} \PYG{n}{tutorial}\PYG{o}{/}\PYG{n}{getting\PYGZus{}started}\PYG{o}{.}\PYG{n}{rst} \PYG{n}{sampledoc}\PYG{o}{/}
\PYG{n}{home}\PYG{p}{:}\PYG{o}{\PYGZti{}}\PYG{o}{/}\PYG{n}{tmp}\PYG{o}{\PYGZgt{}} \PYG{n}{cp} \PYG{n}{tutorial}\PYG{o}{/}\PYG{n}{\PYGZus{}static}\PYG{o}{/}\PYG{n}{basic\PYGZus{}screenshot}\PYG{o}{.}\PYG{n}{png} \PYG{n}{sampledoc}\PYG{o}{/}\PYG{n}{\PYGZus{}static}\PYG{o}{/}
\end{sphinxVerbatim}

The last step is to modify \sphinxcode{\sphinxupquote{index.rst}} to include the
\sphinxcode{\sphinxupquote{getting\_started.rst}} file (be careful with the indentation, the
“g” in “getting\_started” should line up with the ‘:’ in \sphinxcode{\sphinxupquote{:maxdepth}}:

\begin{sphinxVerbatim}[commandchars=\\\{\}]
\PYG{n}{Contents}\PYG{p}{:}

\PYG{o}{.}\PYG{o}{.} \PYG{n}{toctree}\PYG{p}{:}\PYG{p}{:}
   \PYG{p}{:}\PYG{n}{maxdepth}\PYG{p}{:} \PYG{l+m+mi}{2}

   \PYG{n}{getting\PYGZus{}started}\PYG{o}{.}\PYG{n}{rst}
\end{sphinxVerbatim}

and then rebuild the docs:

\begin{sphinxVerbatim}[commandchars=\\\{\}]
\PYG{n}{cd} \PYG{n}{sampledoc}
\PYG{n}{make} \PYG{n}{html}
\end{sphinxVerbatim}

When you reload the page by refreshing your browser pointing to
\sphinxcode{\sphinxupquote{\_build/html/index.html}}, you should see a link to the
“Getting Started” docs, and in there this page with the screenshot.
\sphinxtitleref{Voila!}

We can also use the image directive in \sphinxcode{\sphinxupquote{index.rst}} to include to the screenshot above
with:

\begin{sphinxVerbatim}[commandchars=\\\{\}]
\PYG{o}{.}\PYG{o}{.} \PYG{n}{image}\PYG{p}{:}\PYG{p}{:}
   \PYG{n}{\PYGZus{}static}\PYG{o}{/}\PYG{n}{basic\PYGZus{}screenshot}\PYG{o}{.}\PYG{n}{png}
\end{sphinxVerbatim}

Next we’ll customize the look and feel of our site to give it a logo,
some custom css, and update the navigation panels to look more like
the \sphinxhref{http://sphinx.pocoo.org/}{sphinx} site itself \textendash{} see
\DUrole{xref,std,std-ref}{custom\_look}.


\chapter{Indices and tables}
\label{\detokenize{index:indices-and-tables}}\begin{itemize}
\item {} 
\DUrole{xref,std,std-ref}{genindex}

\item {} 
\DUrole{xref,std,std-ref}{modindex}

\item {} 
\DUrole{xref,std,std-ref}{search}

\end{itemize}



\renewcommand{\indexname}{Index}
\printindex
\end{document}