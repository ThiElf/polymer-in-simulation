%% Generated by Sphinx.
\def\sphinxdocclass{report}
\documentclass[letterpaper,10pt,english]{sphinxmanual}
\ifdefined\pdfpxdimen
   \let\sphinxpxdimen\pdfpxdimen\else\newdimen\sphinxpxdimen
\fi \sphinxpxdimen=.75bp\relax

\PassOptionsToPackage{warn}{textcomp}
\usepackage[utf8]{inputenc}
\ifdefined\DeclareUnicodeCharacter
% support both utf8 and utf8x syntaxes
  \ifdefined\DeclareUnicodeCharacterAsOptional
    \def\sphinxDUC#1{\DeclareUnicodeCharacter{"#1}}
  \else
    \let\sphinxDUC\DeclareUnicodeCharacter
  \fi
  \sphinxDUC{00A0}{\nobreakspace}
  \sphinxDUC{2500}{\sphinxunichar{2500}}
  \sphinxDUC{2502}{\sphinxunichar{2502}}
  \sphinxDUC{2514}{\sphinxunichar{2514}}
  \sphinxDUC{251C}{\sphinxunichar{251C}}
  \sphinxDUC{2572}{\textbackslash}
\fi
\usepackage{cmap}
\usepackage[T1]{fontenc}
\usepackage{amsmath,amssymb,amstext}
\usepackage{babel}



\usepackage{times}
\expandafter\ifx\csname T@LGR\endcsname\relax
\else
% LGR was declared as font encoding
  \substitutefont{LGR}{\rmdefault}{cmr}
  \substitutefont{LGR}{\sfdefault}{cmss}
  \substitutefont{LGR}{\ttdefault}{cmtt}
\fi
\expandafter\ifx\csname T@X2\endcsname\relax
  \expandafter\ifx\csname T@T2A\endcsname\relax
  \else
  % T2A was declared as font encoding
    \substitutefont{T2A}{\rmdefault}{cmr}
    \substitutefont{T2A}{\sfdefault}{cmss}
    \substitutefont{T2A}{\ttdefault}{cmtt}
  \fi
\else
% X2 was declared as font encoding
  \substitutefont{X2}{\rmdefault}{cmr}
  \substitutefont{X2}{\sfdefault}{cmss}
  \substitutefont{X2}{\ttdefault}{cmtt}
\fi


\usepackage[Bjarne]{fncychap}
\usepackage{sphinx}

\fvset{fontsize=\small}
\usepackage{geometry}


% Include hyperref last.
\usepackage{hyperref}
% Fix anchor placement for figures with captions.
\usepackage{hypcap}% it must be loaded after hyperref.
% Set up styles of URL: it should be placed after hyperref.
\urlstyle{same}

\usepackage{sphinxmessages}
\setcounter{tocdepth}{1}



\title{Manual}
\date{Jul 25, 2020}
\release{}
\author{Thiago P. O. Nogueira}
\newcommand{\sphinxlogo}{\vbox{}}
\renewcommand{\releasename}{}
\makeindex
\begin{document}

\pagestyle{empty}
\sphinxmaketitle
\pagestyle{plain}
\sphinxtableofcontents
\pagestyle{normal}
\phantomsection\label{\detokenize{index::doc}}


This is the documentation of a \sphinxtitleref{python} class for generating polymer simulations LAMMPS scripts.

The simulations two kind of systems are generated by this class. One is the simulation of polymer chains
diffusing in a matrix of moving spheres, and the other kind is the simulation of polymer chains diffusing in empty space.

This class can be accesed and downloaded in this \sphinxhref{https://github.com/ThiElf/polymer-in-simulation}{link}.


\chapter{User documentation}
\label{\detokenize{index:user-documentation}}

\section{Getting started}
\label{\detokenize{getting_started:getting-started}}\label{\detokenize{getting_started:id1}}\label{\detokenize{getting_started::doc}}

\subsection{Installing NumPy}
\label{\detokenize{getting_started:installing-numpy}}\label{\detokenize{getting_started:id2}}
To be able to use this class, first one must have NumPy library installed, which is of keen importance to this class.
If you are using Anaconda, you are safe to proceed. If not, this library can be installed by doing on you’re bash:

\begin{sphinxVerbatim}[commandchars=\\\{\}]
\PYG{n}{pip3} \PYG{n}{install} \PYG{n}{numpy}
\end{sphinxVerbatim}


\subsection{Installing fortran}
\label{\detokenize{getting_started:installing-fortran}}\label{\detokenize{getting_started:id3}}
This class is a result of a python and a fortran joint code. To proceed, we need to install gfortran to be able
to fully execute this class. To install gfortran on Linux systems, execute the bellow command on you’re bash:

\begin{sphinxVerbatim}[commandchars=\\\{\}]
\PYG{n}{sudo} \PYG{n}{apt}\PYG{o}{\PYGZhy{}}\PYG{n}{get} \PYG{n}{install} \PYG{n}{gfortran}
\end{sphinxVerbatim}


\subsection{Installing LAMMPS}
\label{\detokenize{getting_started:installing-lammps}}\label{\detokenize{getting_started:id4}}
Since this class generates LAMMPS scripts for polymer simulations, you must have LAMMPS installed on you’re workstation.
To do this, please, access this \sphinxhref{https://lammps.sandia.gov/}{link}.


\section{Class description}
\label{\detokenize{class_description:module-lammps_generator}}\label{\detokenize{class_description:id1}}\label{\detokenize{class_description:class-description}}\label{\detokenize{class_description::doc}}\index{lammps\_generator (module)@\spxentry{lammps\_generator}\spxextra{module}}\index{PolymerSimulation (class in lammps\_generator)@\spxentry{PolymerSimulation}\spxextra{class in lammps\_generator}}

\begin{fulllineitems}
\phantomsection\label{\detokenize{class_description:lammps_generator.PolymerSimulation}}\pysiglinewithargsret{\sphinxbfcode{\sphinxupquote{class }}\sphinxcode{\sphinxupquote{lammps\_generator.}}\sphinxbfcode{\sphinxupquote{PolymerSimulation}}}{\emph{system\_parameters}, \emph{pathto}}{}
This is the class PolymerSimulation. After specifing the needed system parameters,
it will execute the chain.f, or chain\_alone.f, code for the polymers input generation,
and then it will write the LAMMPS inputs for running. chain.f is for the case of mixing
polymers with obstacles, and the chain\_alone.f is for when one wants only to simulate
polymers in empty space.
\begin{quote}\begin{description}
\item[{Parameters}] \leavevmode\begin{itemize}
\item {} 
\sphinxstyleliteralstrong{\sphinxupquote{system\_parameters}} \textendash{} A dictionary with important system parameters.

\item {} 
\sphinxstyleliteralstrong{\sphinxupquote{pathto}} \textendash{} A path to the class directory location.

\end{itemize}

\end{description}\end{quote}

\end{fulllineitems}



\section{System parameters}
\label{\detokenize{input_code:system-parameters}}\label{\detokenize{input_code:id1}}\label{\detokenize{input_code::doc}}

\section{Output code}
\label{\detokenize{output_code:output-code}}\label{\detokenize{output_code::doc}}

\chapter{Indices and tables}
\label{\detokenize{index:indices-and-tables}}\begin{itemize}
\item {} 
\DUrole{xref,std,std-ref}{genindex}

\item {} 
\DUrole{xref,std,std-ref}{modindex}

\item {} 
\DUrole{xref,std,std-ref}{search}

\end{itemize}


\renewcommand{\indexname}{Python Module Index}
\begin{sphinxtheindex}
\let\bigletter\sphinxstyleindexlettergroup
\bigletter{l}
\item\relax\sphinxstyleindexentry{lammps\_generator}\sphinxstyleindexpageref{class_description:\detokenize{module-lammps_generator}}
\end{sphinxtheindex}

\renewcommand{\indexname}{Index}
\printindex
\end{document}